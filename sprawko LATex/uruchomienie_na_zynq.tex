\chapter{Uruchomienie projektu na platformie Digilent Zybo}

Aby uruchomić na platformie Zybo program  wykorzystujący peryferium zaprojektowane wcześniej, należało utworzyć nowy projekt.
Schemat blokowy projektu przedstawiono na rysunku \ref{design_block_diagram}

\begin{figure}[H]
	\centering
	\includegraphics[width = 0.75 \paperwidth]{images/zynq_block_diagram}
	\caption{Schemat blokowy wartsty top projektu}
	\label{design_block_diagram}
\end{figure} 

Wykorzystanie zasobów układu FPGA przedstawiono na rysunku \ref{utilization_plot}. Jest to projekt z peryferium w wersji zredukowanej, czyli znacznie pomniejszonej. Po podpięciu pełnej wersji peryferium, zasoby użyte w projekcie wzrosły. Przedstawiono je na rysunku \ref{utilization_plot_extended}.

\begin{figure}[H]
	\centering
	\includegraphics[width = 0.5 \paperwidth]{images/utilization.png}
	\caption{Wykres zużytych zasobów w wersji zredukowanej}
	\label{utilization_plot}
\end{figure} 

\begin{figure}[H]
	\centering
	\includegraphics[width = 0.5 \paperwidth]{images/utilization_extended.png}
	\caption{Wykres zużytych zasobów w wersji pełnej}
	\label{utilization_plot_extended}
\end{figure} 


